\documentclass[a4,11pt]{aleph-notas}
% Se puede ver la documentación aquí: 
% https://github.com/alephsub0/LaTeX_aleph-notas

% -- Paquetes adicionales 
\usepackage{enumitem}
\usepackage{array,booktabs,multirow,makecell}
\usepackage{colortbl}
\usepackage{longtable}
\usepackage{url}

% -- Datos 
\institucion{Sociedad Ecuatoriana de Estadística}
\asignatura{Fundamentos de Machine Learning}
\tema{Cronograma y actividades}
\autor{Andrés Merino}
\fecha{Febrero 2026}

\logouno[0.14\textwidth]{Logos/logoSEE}
\definecolor{colortext}{HTML}{008dc3}
\definecolor{colordef}{HTML}{008dc3}
\fuente{montserrat}


% -- Comandos para tablas
\newcolumntype{C}[1]{>{\hspace{0pt}\centering\arraybackslash}p{#1}}
\newcolumntype{L}[1]{>{\raggedright\arraybackslash}p{#1}}

\definecolor{verde}{RGB}{0, 255, 127}
\definecolor{celeste}{RGB}{68,195,218}

\begin{document}
\addtolength{\headheight}{1.8\baselineskip}
\addtolength{\voffset}{-1.5\baselineskip}

\encabezado

%%%%%%%%%%%%%%%%%%%%%%%%%%%%%%%%%%%%%%%%
\section{Resultados de aprendizaje} 
%%%%%%%%%%%%%%%%%%%%%%%%%%%%%%%%%%%%%%%%

\begin{itemize}[leftmargin=*]
\item 
    \textbf{RdA 1:} Plantear los conceptos fundamentales del aprendizaje automático.
\item 
    \textbf{RdA 2:} Aplicar modelos de aprendizaje automático supervisado y no supervisado.
\item
    \textbf{RdA 3:} Resolver problemas prácticos mediante el uso de modelos de aprendizaje automático.
\end{itemize}

%%%%%%%%%%%%%%%%%%%%%%%%%%%%%%%%%%%%%%%%
\section{Contenidos generales} 
%%%%%%%%%%%%%%%%%%%%%%%%%%%%%%%%%%%%%%%%

\begin{itemize}
\item 
    Introducción al Aprendizaje Automático: conceptos básicos, flujo de trabajo, tipos de aprendizaje y métricas de distancia.
\item 
    Preprocesamiento de Datos: transformación y reducción de dimensionalidad (PCA).
\item 
    Métodos de Evaluación y Validación de Modelos: partición entrenamiento–prueba, validación cruzada y sobreajuste.
\item 
    Aprendizaje No Supervisado: fundamentos, métricas de similitud, clustering jerárquico y k-means.
\item 
    Aprendizaje Supervisado: métricas, k-NN, SVM, redes neuronales, árboles de decisión y métodos de ensamble.
\item 
    Ajuste y Optimización de Modelos: búsqueda de hiperparámetros, guardado y carga de modelos.
\end{itemize}


%%%%%%%%%%%%%%%%%%%%%%%%%%%%%%%%%%%%%%%%
\section{Actividades de evaluación} 
%%%%%%%%%%%%%%%%%%%%%%%%%%%%%%%%%%%%%%%%


\begin{itemize}[leftmargin=*]
    \item \textbf{Reto 1 (40\%):} Consistirá en la aplicación de modelos de aprendizaje no supervisado en un caso práctico. Los estudiantes deberán analizar los resultados y proponer mejoras basadas en métricas de rendimiento.
    \item \textbf{Reto 2 (60\%):} Consistirá en la aplicación de modelos de aprendizaje supervisado en un escenario del mundo real, ajustando los modelos para maximizar su precisión y eficiencia mediante el ajuste de hiperparámetros y regularización.
\end{itemize}

\newpage
%%%%%%%%%%%%%%%%%%%%%%%%%%%%%%%%%%%%%%%%
\section{Cronograma de Desarrollo del Curso} 
%%%%%%%%%%%%%%%%%%%%%%%%%%%%%%%%%%%%%%%%

\begin{center}\small
\setlength{\extrarowheight}{0ex}
\setlength{\belowrulesep}{.6ex}
\begin{longtable}{cccL{12cm}}
    \toprule
    &&\thead{Fecha}&\thead{Detalle de contenido}  \\
    \midrule
  \endfirsthead
    \multicolumn{4}{l}{\footnotesize \ldots viene de la página anterior}\\
    \toprule
    &&\thead{Fecha}&\thead{Detalle de contenido}  \\
    \midrule
  \endhead
        \bottomrule  \multicolumn{4}{r}{\footnotesize Continúa en la siguiente página\ldots}
  \endfoot
        \bottomrule
  \endlastfoot
1	&	1	&	18-feb	&	Conceptos básicos del Aprendizaje Automático \\	
	&	2	&	19-feb	&	Preparación de datos: normalización y PCA	\\	\midrule
2	&	3	&	23-feb	&	Conjuntos de entrenamiento y prueba; validación cruzada	\\		
	&	4	&	24-feb	&	Aprendizaje no supervisado: métricas e interpretación \\	
	&	5	&	25-feb	&	Agrupamiento k-Means y Clustering jerárquico	\\ 
	&	6	&	26-feb	&	Aprendizaje supervisado: métricas y k-NN	\\	\midrule
3	&	7	&	02-mar	&	SVM	y Árboles de decisión\\	
	&	8	&	03-mar	&	Redes neuronales: perceptrón y perceptrón multicapa	\\		
	&	9	&	04-mar	&	Ensamble de modelos	 y Optimización de hiperparámetros	\\		
\end{longtable}
\end{center}

\end{document} 