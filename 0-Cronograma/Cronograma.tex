\documentclass[a4,11pt]{aleph-notas}
% Se puede ver la documentación aquí: 
% https://github.com/alephsub0/LaTeX_aleph-notas

% -- Paquetes adicionales 
\usepackage{enumitem}
\usepackage{array,booktabs,multirow,makecell}
\usepackage{colortbl}
\usepackage{longtable}
\usepackage{url}
\usepackage{pdflscape}

% -- Datos 
\institucion{Sociedad Ecuatoriana de Estadística}
\asignatura{Fundamentos de Machine Learning}
\tema{Cronograma y actividades}
\autor{Andrés Merino}
\fecha{Febrero 2026}

\logouno[0.14\textwidth]{Logos/logoSEE}
\definecolor{colortext}{HTML}{008dc3}
\definecolor{colordef}{HTML}{008dc3}
\fuente{montserrat}


% -- Comandos para tablas
\newcolumntype{C}[1]{>{\hspace{0pt}\centering\arraybackslash}p{#1}}
\newcolumntype{L}[1]{>{\raggedright\arraybackslash}p{#1}}

\definecolor{verde}{RGB}{0, 255, 127}
\definecolor{celeste}{RGB}{68,195,218}

\begin{document}
\addtolength{\headheight}{1.8\baselineskip}
\addtolength{\voffset}{-1.5\baselineskip}

\encabezado

%%%%%%%%%%%%%%%%%%%%%%%%%%%%%%%%%%%%%%%%
\section{Resultados de aprendizaje} 
%%%%%%%%%%%%%%%%%%%%%%%%%%%%%%%%%%%%%%%%

\begin{itemize}[leftmargin=*]
\item 
    \textbf{RdA 1:} Plantear los conceptos fundamentales del aprendizaje automático.
\item 
    \textbf{RdA 2:} Aplicar modelos de aprendizaje automático supervisado y no supervisado.
\item
    \textbf{RdA 3:} Resolver problemas prácticos mediante el uso de modelos de aprendizaje automático.
\end{itemize}

%%%%%%%%%%%%%%%%%%%%%%%%%%%%%%%%%%%%%%%%
\section{Contenidos generales} 
%%%%%%%%%%%%%%%%%%%%%%%%%%%%%%%%%%%%%%%%

\begin{itemize}
\item 
    Introducción al Aprendizaje Automático: conceptos básicos, flujo de trabajo, tipos de aprendizaje y métricas de distancia.
\item 
    Preprocesamiento de Datos: transformación y reducción de dimensionalidad (PCA).
\item 
    Métodos de Evaluación y Validación de Modelos: partición entrenamiento–prueba, validación cruzada y sobreajuste.
\item 
    Aprendizaje No Supervisado: fundamentos, métricas de similitud, clustering jerárquico y k-means.
\item 
    Aprendizaje Supervisado: métricas, k-NN, SVM, redes neuronales, árboles de decisión y métodos de ensamble.
\item 
    Ajuste y Optimización de Modelos: búsqueda de hiperparámetros, guardado y carga de modelos.
\end{itemize}


%%%%%%%%%%%%%%%%%%%%%%%%%%%%%%%%%%%%%%%%
\section{Actividades de evaluación} 
%%%%%%%%%%%%%%%%%%%%%%%%%%%%%%%%%%%%%%%%


\begin{itemize}[leftmargin=*]
    \item \textbf{Reto 1 (40\%):} Consistirá en la aplicación de modelos de aprendizaje no supervisado en un caso práctico. Los estudiantes deberán analizar los resultados y proponer mejoras basadas en métricas de rendimiento.
    \item \textbf{Reto 2 (60\%):} Consistirá en la aplicación de modelos de aprendizaje supervisado en un escenario del mundo real, ajustando los modelos para maximizar su precisión y eficiencia mediante el ajuste de hiperparámetros y regularización.
\end{itemize}


\begin{landscape}
%%%%%%%%%%%%%%%%%%%%%%%%%%%%%%%%%%%%%%%%
\section{Cronograma de Desarrollo del Curso} 
%%%%%%%%%%%%%%%%%%%%%%%%%%%%%%%%%%%%%%%%

\begin{center}\small
\setlength{\extrarowheight}{0ex}
\setlength{\belowrulesep}{.6ex}
\begin{longtable}{cccL{12cm}L{7.5cm}}
    \toprule
    &&\thead{Fecha}&\thead{Detalle de contenido} & \thead{Observación} \\
    \midrule
  \endfirsthead
    \multicolumn{5}{l}{\footnotesize \ldots viene de la página anterior}\\
    \toprule
    &&\thead{Fecha}&\thead{Detalle de contenido} & \thead{Observación} \\
    \midrule
  \endhead
        \bottomrule  \multicolumn{5}{r}{\footnotesize Continúa en la siguiente página\ldots}
  \endfoot
        \bottomrule
  \endlastfoot
1	&	1	&	24-nov	&	Introducción al curso	&		\\	
	&	2	&	25-nov	&	Conceptos básicos del Aprendizaje Automático	&	Envío del Reto 1 (Criterio 3.1)	\\	
	&	3	&	26-nov	&	Conjuntos de Entrenamiento y Validación	&		\\	
	&	4	&	27-nov	&	Preparación de los Datos	&		\\ \midrule	
2	&	5	&	01-dic	&	Evaluación de Modelos I	&		\\	
	&	5	&	02-dic	&	Evaluación de Modelos II	&		\\	
	&	6	&	03-dic	&	Reducción de Dimensionalidad y Extracción de Características	&		\\	
	&		&	04-dic	&	Taller de implementación	&		\\ \midrule	\rowcolor{celeste!50}
3	&		&	08-dic	&	Cuestionario en línea 1 (Criterio 1.1)	&	Evaluación	\\	
	&	7	&	09-dic	&	Introducción al Aprendizaje No Supervisado	&	Envío de la Video exp. 1 (Criterio 1.2)	\\	
	&	8	&	10-dic	&	Algoritmos de Agrupamiento Jerárquico	&		\\	
	&	9	&	11-dic	&	Agrupamiento k-Means	&		\\ \midrule	
4	&		&	15-dic	&	Desarrollo del Reto 1	&	Entrega del Reto 1 (Criterio 3.1)	\\	\rowcolor{celeste!50}
	&		&	16-dic	&	Cuestionario en línea 2 (Criterio 1.2); Examen 1 (Criterio 2.1)	&	Evaluación	\\	
	&	10	&	17-dic	&	Algoritmo k-Nearest Neighbors	&	Entrega de la Video exp. 1 (Criterio 1.2)	\\	
	&	11	&	18-dic	&	Máquinas de Vectores Soporte (SVM)	&	Envío de la Video exp. 2 (Criterio 1.3)	\\ \midrule	
	&	11	&	22-dic	&	Ajuste y Optimización de SVM	&	Envío del Reto 2 (Criterio 3.2)	\\	
	&		&	23-dic	&	Taller de implementación	&		\\	\rowcolor{verde!50}
	&		&	24-dic	&		&	Feriado	\\	\rowcolor{verde!50}
	&		&	25-dic	&		&	Feriado	\\ \midrule	
5	&	12	&	05-ene	&	Redes Neuronales: Introducción	&		\\	
	&	13	&	06-ene	&	Perceptrón sigmoide y multiclase	&		\\	
	&	14	&	07-ene	&	Perceptrón multicapa	&		\\	
	&	15	&	08-ene	&	Entrenamiento de redes neuronales	&		\\ \midrule	
6	&		&	12-ene	&	Taller de implementación	& Envío de exposiciones		\\	
	&	16	&	13-ene	&	Árboles de Decisión	&		\\	
	&	17	&	14-ene	&	Bosques Aleatorios	&		\\	
	&		&	15-ene	&	Taller de implementación	&	Entrega de la Video exp. 2 (Criterio 1.3)	\\ \midrule	
7	&		&	19-ene	&	Desarrollo del Reto 2	&		\\	\rowcolor{celeste!50}
	&		&	20-ene	&	Cuestionario en línea 3 (Criterio 1.3); Examen 2 (Criterio 2.2)	&	Evaluación	\\	
	&	18	&	21-ene	&	Evaluación de Modelos: Validación cruzada	&		\\	
	&	18	&	22-ene	&	Búsqueda de Hiperparámetros	&		\\ \midrule	
8	&		&	26-ene	&	Taller de implementación	&	Entrega del Reto 2 (Criterio 3.2)	\\	\rowcolor{celeste!50}
	&		&	27-ene	&	Exposiciones	&	Evaluación	\\	\rowcolor{celeste!50}
	&		&	28-ene	&	Exposiciones	&	Evaluación	\\	
	&		&	29-ene	&	Retroalimentación	&		\\ 
\end{longtable}
\end{center}
\end{landscape}

\end{document} 