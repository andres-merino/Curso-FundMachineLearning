\documentclass[aspectratio=169]{beamer}
% -- Formato
\input{formatoBeamer}

% -- Paquetes adicionales
\usepackage{enumitem}
\usepackage{multicol} 
\usepackage{tikz}
\usepackage{qrcode}
\usepackage{fontawesome5}
% -- Comandos extra


% -- Datos
\title{Introducción práctica al Machine Learning con Python}
\author{Andrés Merino T.}
\institute{}
\date{Febrero 2026}



%%%%%%%%%%%%%%%%%%%%%%%%%%%%%%%%%%%%%%%%
\begin{document}

%%%%%%%%%%%%%%%%%%%%%%%%%%%%%%%%%%%%%%%%
% -- Página de título
\fondo{inicio}
\begin{frame}[plain]
\addtocounter{framenumber}{-1}
    \titlepage
\end{frame}

\fondo{blanco}


%%%%%%%%%%%%%%%%%%%%%%%%%%%%%%%%%%%%%%%%
\fondo{celeste}
\section{Objetivo y agenda}
\fondo{blanco}
%%%%%%%%%%%%%%%%%%%%%%%%%%%%%%%%%%%%%%%%
\begin{frame}{Objetivo y agenda}
\textbf{Objetivo:} comprender el flujo mínimo para aplicar Machine Learning en Python y reconocer decisiones clave (tipo de problema, datos, validación y métricas).

\vspace{0.6em}
\textbf{Agenda}
\begin{itemize}
  \item ¿Qué es Machine Learning y cuándo usarlo?
  \item Supervisado vs no supervisado
  \item Métricas y validación
  \item Flujo de trabajo en ML
  \item Demo práctica en Python
\end{itemize}
\end{frame}

%%%%%%%%%%%%%%%%%%%%%%%%%%%%%%%%%%%%%%%%
\fondo{celeste}
\section{¿Qué es Machine Learning?}
\fondo{blanco}
%%%%%%%%%%%%%%%%%%%%%%%%%%%%%%%%%%%%%%%%

\begin{frame}{¿Qué es Machine Learning?}
\begin{itemize}
    \item El Aprendizaje Automático (Machine Learning) es una subdisciplina de la Inteligencia Artificial que se centra en el desarrollo de algoritmos y modelos que permiten a las máquinas aprender a partir de datos sin ser explícitamente programadas para cada tarea específica.
    \item Tom Mitchell (1997): Se dice que un programa de computadora aprende de la {experiencia $E$} con respecto a alguna clase de {tareas $T$} y medida de {desempeño $P$}, si su desempeño en $T$, medido por $P$, {mejora con la experiencia $E$}.
\end{itemize}

\end{frame}

% --- 3) Ejemplos ---
\begin{frame}{Ejemplos típicos de ML}
\begin{columns}[T,onlytextwidth]
\column{0.48\textwidth}
\textbf{Clasificación}
\begin{itemize}
  \item spam / no spam
  \item fraude / no fraude
  \item aprobación / reprobación
\end{itemize}

\column{0.48\textwidth}
\textbf{Regresión}
\begin{itemize}
  \item demanda
  \item precio
  \item tiempo estimado
\end{itemize}
\end{columns}

\vspace{0.7em}
\textbf{No supervisado}
\begin{itemize}
  \item segmentación (clustering), reducción de dimensión, detección de anomalías
\end{itemize}
\end{frame}

% --- 4) Cuándo usar / no usar ---
\begin{frame}{¿Cuándo usar (y cuándo no usar) ML?}
\textbf{Usar ML cuando:}
\begin{itemize}
  \item el patrón es complejo y reglas manuales no escalan,
  \item hay datos suficientes y representativos,
  \item el objetivo y el costo del error están claros.
\end{itemize}

\vspace{0.8em}
\textbf{Evitar ML si:}
\begin{itemize}
  \item una regla simple resuelve la mayor parte del caso,
  \item los datos son escasos/inestables o la etiqueta es poco confiable,
  \item no existe plan de evaluación y monitoreo.
\end{itemize}
\end{frame}

% --- 5) Conceptos mínimos ---
\begin{frame}{Conceptos mínimos (para no perderse)}
\begin{itemize}
  \item \textbf{Feature (variable):} columna de entrada ($X$).
  \item \textbf{Etiqueta (target):} lo que se quiere predecir ($y$).
  \item \textbf{Modelo:} algoritmo + parámetros aprendidos.
  \item \textbf{Entrenar:} ajustar el modelo con datos de entrenamiento.
  \item \textbf{Predecir:} estimar $y$ para datos nuevos.
  \item \textbf{Generalización:} desempeño en datos no vistos.
\end{itemize}
\end{frame}


%%%%%%%%%%%%%%%%%%%%%%%%%%%%%%%%%%%%%%%%
\fondo{celeste}
\section{Supervisado vs no supervisado}
\fondo{blanco}
%%%%%%%%%%%%%%%%%%%%%%%%%%%%%%%%%%%%%%%%

% --- 6) Supervisado vs no supervisado ---
\begin{frame}{Supervisado vs no supervisado}
\begin{columns}[T,onlytextwidth]
\column{0.48\textwidth}
\textbf{Aprendizaje supervisado}
\begin{itemize}
  \item Datos: $(X, y)$
  \item Objetivo: predecir $y$
  \item Tareas: clasificación, regresión
\end{itemize}

\column{0.48\textwidth}
\textbf{Aprendizaje no supervisado}
\begin{itemize}
  \item Datos: $X$
  \item Objetivo: descubrir estructura
  \item Tareas: clustering, PCA, anomalías
\end{itemize}
\end{columns}

\vspace{0.6em}
\textbf{Pregunta guía:} ¿tengo una etiqueta confiable para aprender?
\end{frame}

% --- 7) Ejemplos supervisado ---
\begin{frame}{Supervisado: clasificación vs regresión}
\begin{columns}[T,onlytextwidth]
\column{0.48\textwidth}
\textbf{Clasificación}
\begin{itemize}
  \item salida discreta (clases)
  \item ejemplo: \textit{riesgo alto/medio/bajo}
\end{itemize}
\textbf{Salida:} ${y}\in\{0,1,\dots\}$

\column{0.48\textwidth}
\textbf{Regresión}
\begin{itemize}
  \item salida continua
  \item ejemplo: \textit{monto esperado}
\end{itemize}
\textbf{Salida:} ${y}\in\mathbb{R}$
\end{columns}
\end{frame}

\begin{frame}{Supervisado: idea matemática}
\begin{itemize}
  \item Disponemos de un conjunto de datos etiquetados:
  \[
    \mathcal{D}=\{(x_i, y_i)\}_{i=1}^n,
  \]
  donde \(x_i\) son los datos de entrada (variables, features) y \(y_i\) la etiqueta o salida observada.
  
  \item Suponemos que existe una función (desconocida) \(f\) tal que:
  \[
    y = f(x).
  \]

  \item El objetivo del aprendizaje supervisado es encontrar una función aproximante \(\tilde{f}\) tal que:
  \[
    \tilde{f}(x) \approx y.
  \]
\end{itemize}
\end{frame}


%%%%%%%%%%%%%%%%%%%%%%%%%%%%%%%%%%%%%%%%
\fondo{celeste}
\section{Métricas y validación}
\fondo{blanco}
%%%%%%%%%%%%%%%%%%%%%%%%%%%%%%%%%%%%%%%%

\begin{frame}{Métricas y validación}
\begin{itemize}
  \item Una vez obtenida la función aproximante \(\tilde{f}\), surge la pregunta:
  \[
    \text{¿Qué tan bien aproxima } \tilde{f}(x) \text{ a } y?
  \]

  \item Evaluamos el desempeño mediante una \textbf{métrica de error} en datos observados:
  \begin{itemize}
    \item Clasificación: accuracy, precision, recall, F1
    \item Regresión: MAE, MSE, RMSE
  \end{itemize}
\end{itemize}
\end{frame}

\begin{frame}
\begin{itemize}
  \item \textbf{Sobreajuste (overfitting):}
  \begin{itemize}
    \item \(\tilde{f}\) se ajusta demasiado a los datos.
    \item Aprende el ruido en lugar del patrón general.
    \item Resultado: muy buen desempeño en datos conocidos, mal desempeño en datos nuevos.
  \end{itemize}

  \item \textbf{Motivación del split train/test:}
  \begin{itemize}
    \item Dividimos el conjunto de datos conocidos en dos grupos: train/test.
    \item Entrenamos \(\tilde{f}\) con \textbf{train}.
    \item Evaluamos qué tan bien generaliza usando \textbf{test} (datos conocidos pero no vistos).
    \item El conjunto test actúa como un ``examen final'' de la función.
  \end{itemize}
\end{itemize}
\end{frame}

%%%%%%%%%%%%%%%%%%%%%%%%%%%%%%%%%%%%%%%%
\fondo{celeste}
\section{Flujo de trabajo en ML}
\fondo{blanco}
%%%%%%%%%%%%%%%%%%%%%%%%%%%%%%%%%%%%%%%%

\begin{frame}{Flujo de trabajo en Machine Learning}
\begin{enumerate}
  \item \textbf{Entender el problema (contexto/negocio)}
  \begin{itemize}
    \item ¿Qué decisión se quiere apoyar? ¿Cuál es el objetivo?
    \item ¿Qué significa un error? ¿Cuál es su costo?
  \end{itemize}

  \item \textbf{Entender los datos}
  \begin{itemize}
    \item ¿Qué variables existen? ¿Cómo se generan?
    \item Revisar calidad: nulos, ruido, sesgos, cobertura.
  \end{itemize}

  \item \textbf{Definir el problema de ML}
  \begin{itemize}
    \item Tipo: clasificación, regresión o no supervisado.
    \item Variable objetivo, unidad de análisis y horizonte.
  \end{itemize}
\end{enumerate}
\end{frame}

\begin{frame}{Flujo de trabajo en Machine Learning}
\begin{enumerate}[start=4]
  \item \textbf{Preparar los datos}
  \begin{itemize}
    \item Limpieza, selección/creación de variables, escalado/categóricas.
    \item Separar en \textbf{train} y \textbf{test}.
  \end{itemize}

  \item \textbf{Entrenar el modelo}
  \begin{itemize}
    \item Empezar con un baseline simple.
    \item Ajustar el modelo con datos de entrenamiento.
  \end{itemize}

  \item \textbf{Medir desempeño}
  \begin{itemize}
    \item Evaluar en entrenamiento (diagnóstico) y en test (generalización).
    \item Usar métricas adecuadas al problema.
  \end{itemize}
\end{enumerate}
\end{frame}

\begin{frame}{Flujo de trabajo en Machine Learning}
\begin{enumerate}[start=7]
  \item \textbf{Iterar y mejorar}
  \begin{itemize}
    \item Mejorar datos, variables, modelo o validación.
    \item Comparar resultados y documentar.
  \end{itemize}
\end{enumerate}
\end{frame}


%%%%%%%%%%%%%%%%%%%%%%%%%%%%%%%%%%%%%%%%
\fondo{celeste}
\section{Demo práctica en Python}
\fondo{blanco}
%%%%%%%%%%%%%%%%%%%%%%%%%%%%%%%%%%%%%%%%

\begin{frame}
\begin{center}\color{azul}
    \qrcode[height=5.5cm]{https://github.com/andres-merino/Curso-FundMachineLearning/blob/main/1-Webinar/Webinar.ipynb}
    \\[5mm]
    Cuaderno de Jupyter
\end{center}
\end{frame}


%%%%%%%%%%%%%%%%%%%%%%%%%%%%%%%%%%%%%%%%
%% Página final
\fondo{final}
\begin{frame}[plain]
\begin{center}
    \color{azul}
    
    \vspace{1.5cm}
    {\Huge\textbf{Gracias}}
    \vspace{2mm}
    
    \begin{tabular}{ccc}
    \colorbox{white}{\textcolor{azul}{\qrcode[height=3cm]{https://linktr.ee/aemerinot}}}
    &
    \hspace{1cm}
    &
    \colorbox{white}{\textcolor{azul}{\qrcode[height=3cm]{https://andres-merino.github.io/Curso-FundMachineLearning/1-Webinar/Webinar.pdf}}}
    \\[16mm]
    \LARGE \faLinkedin\hspace{5mm}\faGithub
    &&
    Presentación
\end{tabular}
\\
    \vspace{2mm}
    \textbf{Contacto:} aemerinot@gmail.com
\end{center}
\end{frame}

\begin{frame}[plain]
\begin{center}
    \includegraphics[width=7cm]{1-Webinar/fig01.jpeg}
\end{center}
\end{frame}

\end{document}